\documentclass[10pt,a4paper]{report}
\usepackage[latin1]{inputenc}
\usepackage{amsmath}
\usepackage{amsfonts}
\usepackage{amssymb}
\usepackage{graphicx}
\usepackage{hyperref}

\newcommand{\BCDS}{\href{http://arxiv.org/abs/1405.1096}{Budanur \emph{et al.} (2005)}}
\newcommand{\KasTre}{\href{http://eprints.maths.ox.ac.uk/1194/1/NA-03-14.pdf}{Kassam \& Trefethen 2003}}
\newcommand{\ssp}{a}             % state space point
\newcommand{\sspC}{a}             % complex valued state space point
\newcommand{\man}{\ensuremath{{\cal M}}}             % manifold
\newcommand{\pSRed}{\ensuremath{\hat{\cal M}}} % reduced state space
\newcommand{\sspRed}{\ensuremath{\hat{\ssp}}}    % reduced state space point, experiment
\newcommand{\sspCRed}{\ensuremath{\hat{\sspC}}}    % reduced state space point, experiment
\newcommand{\vel}{\ensuremath{v}}
\newcommand{\velRed}{\ensuremath{\hat{\vel}}}    % ES reduced state space velocity
\newcommand{\slicep}{\ensuremath{\hat{\ssp}'}}   % slice-fixing point, experimental
\newcommand{\Fourier}{\ensuremath{{\cal F}}}
\newcommand{\Fu}{\ensuremath{\tilde{u}}}
\newcommand{\zeit}{\ensuremath{\tau}}
\newcommand{\zeitRed}{\ensuremath{\hat{\zeit}}}
\newcommand{\KS}{Kuramoto-Sivashisnky}
\newcommand{\groupTan}{\ensuremath{t}}
\newcommand{\gSpace}{\ensuremath{\theta}}

\title{Adapting ETDRK4 (Kassam \& Trefethen 2003) for integrating 
	   \KS\ equation in the first Fourier mode slice}
\author{Nazmi Burak Budanur}

\begin{document}

\maketitle

\section*{Introduction}
In \BCDS, we adapted Exponential time differencing fourth-order Runge-Kutta
(ETDRK4) method for integrating 1D \KS\ system within the first Fourier mode
slice hyperplane. In this short document, we present details of this
implementation which are not immediately related to the content of \BCDS.

\section*{Implementation}

ETDRK4 method is used to solve nonlinear ODEs of the following form:
\begin{equation}
	\dot{\sspC} = \vel (\sspC) = L \sspC + N(\sspC) \, ,
	\label{e-ODE}
\end{equation}	
where $\sspC \in \man$ is a complex valued state space vector, $L$ is a matrix
and $N(\sspC)$ is a nonlinear function. \KS\ equation is of this form with
$L_{kl} = (q_k^2 - q_k^4) \delta_{kl}$ (diagonal), $N_k (a) = - i \frac{q_k}{2}
\sum_{m=-\infty}^{\infty} \Fu_m \Fu_{k - m}$ and $\sspC = (\Fu_1, \Fu_2, \Fu_3,
\ldots)^T\,\mbox{where, } \Fu (\zeit) = \Fourier \{u(x, \zeit)\} $. Numerical
integration of \KS\ equation is explained in detail, with a Matlab code in 
\KasTre . This is the code that our implementation \texttt{ksETDRK4red.m} is
based on. In the complex representation, dynamical equations with respect to
the first Fourier mode slice time yields
\begin{eqnarray}
	\frac{d \sspRed}{d \zeitRed}  &=& Re[\sspRed_1] \vel(\sspRed)
	- Im[\vel(\sspRed)_1] \, \groupTan(\sspRed) \label{e-ffSliceStatsp}
	\\
	\frac{d \gSpace}{d \zeitRed} &=& Im[\vel(\sspRed)_1] \label{e-ffSlicePhase}
	\,.
\end{eqnarray}
where, $d \zeitRed = Re[\sspRed_1]^{-1} d \zeit$ is the in-slice time,
$\groupTan(\sspRed) = T \sspRed $ is the group tangent. In the complex
representation, $SO(2)$ symmetry becomes $U(1)$, hence $T_{kl} = i \,
k\delta_{kl}$. Linear and nonlinear parts of equation \ref{e-ffSliceStatsp} are
not immediately separable, however, we can add and subtract a constant $\alpha$
to $Re[\sspRed_1]$ in \ref{e-ffSliceStatsp} and rewrite it as follows
\begin{equation}
	\frac{d \sspRed}{d \zeitRed} = \tilde{L} \sspRed + \tilde{N}(\sspRed) .
\end{equation}
where, 
\begin{eqnarray}
	\tilde{L} &=& \alpha L ,
	  \\
	\tilde{N}(\sspRed) &=& N(\sspRed) + (Re[\sspRed_1] - \alpha) \vel (\sspRed)
	  			- Im[\vel(\sspRed)_1] \, \groupTan(\sspRed) .
\end{eqnarray}	
In our implementation, we set the parameter $\alpha$ experimentally to $1$.
\end{document}