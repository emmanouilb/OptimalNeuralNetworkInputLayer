% defs.tex
% $Author: predrag $ $Date: 2007-04-26 08:22:46 -0400 (Thu, 26 Apr 2007) $

%%              Predrag  22 feb 2000
%% from defExam.tex         Predrag   1 may 2006
%%              Predrag  21 may 2000
%%              Carl     30 oct 1997
%% created              Predrag  26 oct 1997

%%%%%%%%%%%%%%% REFERENCING EQUATIONS ETC. %%%%%%%%%%%%%%%%%%%%%%%%%%%%%%%

\newcommand{\rf}     [1] {~\cite{#1}}
\newcommand{\refref} [1] {ref.~\cite{#1}}
\newcommand{\refRef} [1] {Ref.~\cite{#1}}
\newcommand{\refrefs}[1] {refs.~\cite{#1}}
\newcommand{\Refrefs}[1] {Refs.~\cite{#1}}
\newcommand{\refeq}  [1] {(\ref{#1})}
\newcommand{\refeqs} [2]{(\ref{#1}-\ref{#2})}
\newcommand{\refpage}[1] {page~\pageref{#1}}
\newcommand{\reffig} [1] {fig.~\ref{#1}}
\newcommand{\reffigs} [2] {figs.~\ref{#1} and~\ref{#2}}
\newcommand{\refFig} [1] {Fig.~\ref{#1}}
\newcommand{\reftab} [1] {table~\ref{#1}}
\newcommand{\refTab} [1] {Table~\ref{#1}}
\newcommand{\reftabs}[2] {tables~\ref{#1} and~\ref{#2}}
\newcommand{\refrem} [1] {remark~\ref{#1}}
\newcommand{\refsect}[1] {sect.~\ref{#1}}
\newcommand{\refsects}[2] {sects.~\ref{#1} and \ref{#2}}
\newcommand{\refSect}[1] {Sect.~\ref{#1}}
\newcommand{\refexam}[1] {example~\ref{#1}}
\newcommand{\refExam}[1] {Example~\ref{#1}}
\newcommand{\refexer}[1] {exercise~\ref{#1}}
\newcommand{\refExer}[1] {Exercise~\ref{#1}}
\newcommand{\refsolu}[1] {solution~\ref{#1}}

\newcommand{\weblink}[1]{{\tt \href{http://#1}{#1}}}
\newcommand{\HREF}[2]{
          {\href{#1}{#2}}}
\newcommand{\mpArc}[1]{
          {\tt \href{http://www.ma.utexas.edu/mp_arc-bin/mpa?yn=#1}
               {\goodbreak mp\_arc~#1}}}
\newcommand{\arXiv}[1]{
          {\tt \href{http://arXiv.org/abs/#1}{\goodbreak #1}}}


%%%%%%%%%%%%% symbols within math eviron %%%%%%%%%%%%%%%%

\newcommand\stagn{*}            %equilibrium/stagnation point suffix

%%%%%%%%%%%%%%%%%%%%%% final.tex SPECIFIC %%%%%%%%%%%%%%%%%%%%%%%%%%%%%%%
\newcommand{\Exam}{\marginpar{\fbox{\sf
                % $\bullet$
                    Exam question
                % $\bullet$
                  } } }

\newcommand{\Points}[1]{\marginpar{\fbox{\sf
                % $\bullet$
                    #1~point
                % $\bullet$
                  } } }

\newcommand{\Optional}{\marginpar{\fbox{\sf
                % $\bullet$
                    Optional
                % $\bullet$
                  } } }

%%%%%%%%%%%%%%%%%%%%%% COMMENTS IN THE TEXT %%%%%%%%%%%%%%%%%%%%%%%%%%%%%
%% also search the text for lines starting with %%  to
%% locate various internal comments, recent edits etc.
\newcommand{\PC}[1]{\marginpar{
        \renewcommand{\baselinestretch}{0.7}
        \footnotesize #1}
        \renewcommand{\baselinestretch}{1.0}
                }
%\newcommand{\PC}[1]{$\footnotemark\footnotetext{PC: #1}$}
\newcommand{\JH}[1]{$\footnotemark\footnotetext{DL: #1}$}
\newcommand{\PCedit}[1]{{\color{blue}#1}}
\newcommand{\JHedit}[1]{{\color{red}#1}}

\newcommand{\file}[1]{$\footnotemark\footnotetext{{\bf file} #1}$}


%%%%%%%%%%%%%%% EQUATIONS %%%%%%%%%%%%%%%%%%%%%%%%%%%%%%%
\newcommand{\beq}{\begin{equation}}
\newcommand{\continue}{\nonumber \\ }
\newcommand{\nnu}{\nonumber}
\newcommand{\eeq}{\end{equation}}
\newcommand{\ee}[1] {\label{#1} \end{equation}}
\newcommand{\bea}{\begin{eqnarray}}
\newcommand{\ceq}{\nonumber \\ & & }
\newcommand{\eea}{\end{eqnarray}}
\newcommand{\barr}{\begin{array}}
\newcommand{\earr}{\end{array}}

%%%%%%%%%%%%%%% VECTORS, MATRICES %%%%%%%%%%%%%%%%%%%%%%%%%%%%%%%
\newcommand{\MatrixII}[4]{
   \pmatrix{ {#1}  &  {#2} \cr
             {#3}  &  {#4} \cr} }

\newcommand{\MatrixIII}[9]{
   \pmatrix{ {#1}  &  {#2} &  {#3} \cr
             {#4}  &  {#5} &  {#6} \cr
             {#7}  &  {#8} &  {#9} \cr} }

\newcommand{\transpVectorII}[2]{
   \pmatrix{ {#1}  &  {#2}  \cr} }

\newcommand{\VectorII}[2]{
   \pmatrix{ {#1} \cr
             {#2} \cr} }

\newcommand{\VectorIII}[3]{
   \pmatrix{ {#1} \cr
             {#2} \cr
             {#3} \cr} }

\newcommand{\combinatorial}[2]{ {#1 \choose #2}}

%%%%%%%%%%%%%%% NUMBERED ENVIRONMENTS %%%%%%%%%%%%%%%%%%%%%%%%%%%%%%%
%   \FIG{#1}    % \epsfig{file=figs/f_name.ps,width=?cm} ... here
%   {#2}    % short caption text
%   {#3}    % full caption text
%   {#4}    % f-figure-label
%       defined here:
\newcommand{\FIG}[4]{\begin{figure}
              \hspace*{0.10\textwidth}%
              \begin{minipage}[b]{1.00\textwidth}
              \noindent{#1}
              %\centering{#1}
                      \caption[#2]{#3}
                      \label{#4}
              \end{minipage}
              \end{figure} }

%  \SFIG{#1}    % f_name [omit .eps or .pdf]
%       {#2}    % short caption text
%       {#3}    % full caption text
%       {#4}    % f-figure-label
\newcommand{\SFIG}[4]{\begin{figure}
              %\hspace*{-0.10\textwidth}
              \hspace*{0.10\textwidth}
              \begin{minipage}[b]{0.55\textwidth}
                      \caption[#2]{#3}
                      \label{#4} 
              \end{minipage}~~~~~%
              \begin{minipage}[b]{0.40\textwidth}
                      \includegraphics[width=1.00\textwidth]{Fig/#1}
              \end{minipage}
              %\hfill
              \end{figure} }

%%%%%%%%%%%%%%%%%%%%%% QUOTATIONS %%%%%%%%%%%%%%%%%%%%%%%%%%%%%%%%%%%%%%
%
%  the learned/witty quotes at the chapter and section headings
% 
\newsavebox{\bartName}
\newcommand{\bauthor}[1]{\sbox{\bartName}{\parbox{\textwidth}{\vspace*{0.8ex}
       %\hspace*{\fill}
       \small\noindent #1}}}
\newenvironment{bartlett}{\hfill\begin{minipage}[t]{0.65\textwidth}\small}%
{\hspace*{\fill}\nolinebreak[1]\usebox{\bartName}\vspace*{1ex}\end{minipage}}

\newcommand{\Remarks}{
                \section*{\textsf{\textbf{Commentary}}}
                        }

\newtheorem{rmark}{{\small\textsf{\textbf{Remark}}}}%[chapter]
\newcommand{\remark}[2]{
        \begin{rmark}
        {\small\em\noindent {\small\sf \underline{ #1} ~} #2 }
    \end{rmark}
              }

\newtheorem{exerc}{\textsf{\textbf{Exercise}}} %[chapter]
 \newcommand{\exercise}[2]{
        \vskip -13mm
         \noindent
         \begin{exerc}{
\renewcommand{\theenumi}{\alph{enumi}}
\renewcommand{\labelenumi}{\textbf{(\alph{enumi})\ }}
    {\noindent\small
         ~~\textsf{\textbf{#1}} ~
           \slshape\sffamily{#2}  } % \textsl would not work...
    }
         \vskip -1mm 
% removed the line: % \noindent\rule[.1mm]{\linewidth}{.5mm}
         \end{exerc}
                          }

\newcommand{\Exercise}[2]{      %environment for obligatory problems
        \vskip -13mm
        \noindent
        \begin{exerc}{
\renewcommand{\theenumi}{\alph{enumi}}
\renewcommand{\labelenumi}
    {\textsf{\textbf{ (\alph{enumi})\ }}}
        {\noindent
         ~~\textsf{\textbf{\underline{#1}}} ~
           \slshape\sffamily{#2}  } % \textsl would not work...
        }
         \vskip -1mm
        \end{exerc}
                          }
 
\newcommand{\solution}[3]{
         \vskip -4mm
        {\noindent\small
         ~~\textsf{\textbf{  Solution \ref{#1}:   %NUMBER
                ~#2}}         %TITLE 
           \slshape\sffamily{#3}          %TEXT
         }
         \vskip  4mm 
% removed the line: % \noindent\rule[.1mm]{\linewidth}{.5mm}
                        }


%%%%%%%%%%%%%%% Evolution operators, zetas %%%%%%%%%%%%%%%%%%%%%%%%%%%
\newcommand{\etc}{{\em etc.}}
\newcommand{\evOper}{evolution oper\-ator}
\newcommand{\EvOper}{Evolution oper\-ator}
\newcommand{\FPoper}{Perron-Frobenius oper\-ator} % Pesin's ordering
\newcommand{\FP}{Perron-Frobenius}
\newcommand{\jacobian}{Jacobian}        % determinant
\newcommand{\jacobianM}{Jacobian matrix}    % matrix
\newcommand{\jacobianMs}{Jacobian matrices} % matrices
\newcommand{\dzeta}{dyn\-am\-ic\-al zeta func\-tion}
\newcommand{\Dzeta}{Dyn\-am\-ic\-al zeta func\-tion}
\newcommand{\tzeta}{top\-o\-lo\-gi\-cal zeta func\-tion}
\newcommand{\Tzeta}{Top\-o\-lo\-gi\-cal zeta func\-tion}
\newcommand{\qS}{semi\-classical zeta func\-tion}
\newcommand{\Gt}{Gutz\-willer trace formula}
\newcommand{\Fd}{spec\-tral det\-er\-min\-ant}
\newcommand{\fd}{spec\-tral det\-er\-min\-ant}
\newcommand{\FD}{Spec\-tral det\-er\-min\-ant}
\newcommand{\cFd}{class\-ic\-al spec\-tral det\-er\-mi\-nant}
\newcommand{\cycForm}{cycle averaging formula}
\newcommand{\CycForm}{Cycle averaging formula}
\newcommand{\stretchf}{``stretch \&\ fold''}
\newcommand{\Stretchf}{``Stretch \&\ fold''}
\newcommand{\statesp}{state space}
\newcommand{\Statesp}{State space}
\newcommand{\stabmat}{stability matrix}     % stability matrix
\newcommand{\Stabmat}{Stability matrix}     % Stability matrix
% \newcommand{\stabmat}{matrix of variations}   % Arnold, says Vattay

\newcommand{\tildeL}{\ensuremath{\tilde{L}}}
% \newcommand{\J}{\ensuremath{\mathbf{J}}}

%%%%%%%%%%%% SHORTCUTS, project specific %%%%%%%%%%
\newcommand{\po}{periodic orbit}
\newcommand{\Po}{Periodic orbit}
\newcommand{\rpo}{relative periodic orbit}
%   \newcommand{\rpo}{equivariant periodic orbit}
\newcommand{\Rpo}{Relative periodic orbit}
%   \newcommand{\Rpo}{Equivariant periodic orbit}

\newcommand{\eqv}{equilibrium}
\newcommand{\Eqv}{equilibrium}
\newcommand{\eqva}{equilibria}
\newcommand{\Eqva}{Equilibria}
\newcommand{\reqv}{relative equilibrium}
%   \newcommand{\reqv}{equivariant equilibrium}
%   \newcommand{\reqv}{travelling wave}
\newcommand{\Reqv}{Relative equilibrium}
%   \newcommand{\Reqv}{Equivariant equilibrium}
%   \newcommand{\Reqv}{travelling wave}
\newcommand{\reqva}{relative equilibria}
%   \newcommand{\reqva}{equivariant equilibria}
\newcommand{\Reqva}{Relative equilibria}
%   \newcommand{\Reqva}{Equivariant equilibria}

\newcommand{\KS}{Kuramoto-Sivashinsky}
\newcommand{\KSe}{Kuramoto-Sivashinsky equation}
\newcommand{\pCf}{plane Couette flow}
\newcommand{\PCf}{Plane Couette flow}

% if still nameless:
\newcommand{\nameit}{\ensuremath{w-}}
% Ruslan likes:
\newcommand{\EQV}[1]{\ensuremath{E_{#1}}}
% E_0: u = 0 - trivial equilibrium
% E_1,E_2,E_3, for 1,2,3-wave equilibria
\newcommand{\REQV}[2]{\ensuremath{TW_{#1#2}}} % #1 is + or -
% TW_1^{+,-} for 1-wave travelling waves (positive and negative velocity).
\newcommand{\PO}[1]{\ensuremath{PO_{#1}}}
% PO_{period to 2-4 significant digits} - periodic orbits
\newcommand{\RPO}[1]{\ensuremath{RPO_{#1}}}
% RPO_{period to 2-4 significant digits} - relative PO.  We also can use ^{+,-}
% here if necessary to distinguish between members of a reflection-symmetric
% pair.
% PC not happy about ^{+,-} for reflection-symmetric pairs


%%%%%%%%%%%%%%% Sundry symbols within math eviron.: %%%%%%%%%%%%
\newcommand{\DiffConst}{D}
\newcommand{\obser}{a}      % an observable from phase space to R^n
\newcommand{\Obser}{A}      % time integral of an observable
\newcommand{\onefun}{\iota} % the function that returns one no matter what
\newcommand{\defeq}{=}      % the different equal for a definition
\newcommand {\deff}{\stackrel{\rm def}{=}}
\newcommand{\reals}{\mathbb{R}}
\newcommand{\integers}{\mathbb{Z}}
\newcommand{\rationals}{\mathbb{Q}}
\newcommand{\naturals}{\mathbb{N}}
\newcommand{\LieD}{{{\cal L}\!\!\llap{-}\,\,}}  % {{\pound}} % Lie Derivative 
\newcommand{\half}{{\scriptstyle{1\over2}}}
\newcommand{\pde}{\partial}
\renewcommand\Im{{\rm Im\,}}
\renewcommand\Re{{\rm Re\,}}
\renewcommand{\det}{\mbox{\rm det}}
\newcommand{\Det}{\mbox{\rm Det}\, }
\newcommand{\tr}{{\rm tr}\, }
\newcommand{\Tr}{\mbox{Tr}\,}
\newcommand{\sign}[1]{{\rm sign}(#1)}
\newcommand{\msr}{{\mu}}                % measure 
\newcommand{\vol}{{V}}                  % volume of i-th tile
\newcommand{\prpgtr}[1]{\delta\negthinspace\left( {#1} \right)}
\newcommand{\Zqm}{Z_{qm}}           % Gutz-Voros zeta function
\newcommand{\Fqm}{F_{qm}}
\newcommand{\zfct}[1]{\zeta ^{-1}_{#1}}     
\newcommand{\zetatop}{1/\zeta_{\mbox{\footnotesize top}}}
\newcommand{\expct}    [1]{\left\langle {#1} \right\rangle}
\newcommand{\spaceAver}[1]{\left\langle {#1} \right\rangle}
\newcommand{\timeAver} [1]{\overline{#1}}
\newcommand{\pS}{{\cal M}}          % symbol for phase space 
\newcommand{\intM}[1]{{\int_\pS{\!d #1}\:}} %phasespace integral
\newcommand{\Lop}{{\cal L}}    % evolution operator
\newcommand{\Uop}{{\cal U}}    % Koopman operator
\newcommand{\Aop}{{\cal A}}    % evolution generator
\newcommand{\Top}{{\cal T}}    % transfer operator, like in statmech
\newcommand{\jMps}{{\bf J}}    %jacobiam matrix, full phase space
\newcommand{\oneMinJ}[1]{\left|\det\left({\bf 1}-{\bf J}_p^{#1}\right)\right|}
                   %Fredholm det jacobian weight
\newcommand{\Mvar}{{A}}        % matrix of variations
\newcommand{\monodromy}{{\bf J}}   %monodromy matrix, full Poincare cut
\newcommand{\maslov}[1]{{m_{#1}}}  %Maslov index
%\newcommand{\eigenvL}{{s}}    
\newcommand{\eigenvL}{{\nu}}       %This is really eigenvalue, not log!
\newcommand{\inFix}[1]{{\in \mbox{\footnotesize Fix}f^{#1}}}
\newcommand{\inZero}[1]{{\in \mbox{\footnotesize Zero} \, f^{#1} }}
\newcommand{\xzero}[1]{{x_{#1}^{*}}}
\newcommand{\ExpaEig}{\ensuremath{\Lambda}}
%%       optional parameter comes in [\ldots], for example
%%       \newcommand\eigRe[1][ ]{\ensuremath{\mu_{#1}}}
%%       no subscript: \eigRe\
%%       with subscript j: \eigRe[j]
%%
\newcommand{\eigExp}[1][ ]{\ensuremath{\lambda_{#1}}}   % complex eigenexponent
%%      Guckenheimer&Holmes:  lambda = alpha + i beta
%%      Hirsch-Smale:         lambda = a     + i b
%%      Boyce-di Prima:       lambda = mu    + i nu
\newcommand{\eigRe}[1][ ]{\ensuremath{\mu_{#1}}}        % Re eigenexponent
\newcommand{\eigIm}[1][ ]{\ensuremath{\nu_{#1}}}        % Im eigenexponent
\newcommand{\Lyap}{\ensuremath{\lambda}}  %Lyapunov exponent
\newcommand{\jEigvec}[1]{\ensuremath{\mathbf{e}^{(#1)}}} % eigenvector
\newcommand{\orderof}[1]{o(#1)} % Rytis 22mar2005
\newcommand{\PoincS}{\ensuremath{\cal P}}  % symbol for Poincare section
\newcommand{\PoincM}{\ensuremath{P}}       % symbol for Poincare map
\newcommand{\PoincC}{\ensuremath{U}}       % symbol for Poincare constraint function
\newcommand{\matId}{\ensuremath{\mathbf{1}}}          % matrix identity
\newcommand{\pVeloc}{v}         % phase-space velocity 

%%%%%%%%%%%%%%%%%%%%%%%%%%%%%%%%%%%%%%%%%%%%%%%%%%%%%%%%%%%

\newcommand{\eps} {\varepsilon}
\newcommand{\twopi}{{2\pi}}
\newcommand{\avert}[1]{\overline{#1}}
\newcommand{\averx}{\int dx\,}
\newcommand{\derJ}{{d~\over dJ}}

%%%%%%%%%% flows: %%%%%%%%%%%%%%%%%%%%%%%%%%%%
\newcommand\flow[2]{{f^{#1}(#2)}}
\newcommand\xInit{{x_0}}        %initial x
\newcommand\pSpace{x}       % phase space x=(q,p) coordinate
\newcommand{\para}{\parallel}

%%%%%%%%%%%%%%%% periods: %%%%%%%%%%%%%%%%%%%%%%%%%%%%
\newcommand\period[1]{{T_{#1}}}         %continuous cycle period
\newcommand{\cl}[1]{{n_{#1}}}   % discrete length of a cycle, Predrag
\newcommand{\timeSegm}[1]{{\tau_{#1}}}      %billiard segment time period
\newcommand{\dz}{{\delta z}}

%%%%%%%%%%%%%%% symbolic dynamics %%%%%%%%%%%%%%%%%%%%%%%%%%%%%%%%%%
\newcommand{\MarkGraph}{Markov graph}
\newcommand{\admissible}{admissible}
\newcommand{\inadmissible}{inadmissible}
\newcommand{\cycle}[1]{{$\overline{#1}$}}
\newcommand{\cycpt}{_{p,m}}
\newcommand\sumprime{\mathop{{\sum}'}}
\newcommand{\pseudos}{{p_1+p_2+\dots+p_k}}
\newcommand{\block}[1]{#1}
\newcommand{\Ksym}[1]{\sigma_{#1}}
\newcommand{\Ssym}[1]{s_{#1}}

